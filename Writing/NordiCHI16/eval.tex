For evaluation, we conducted individual interviews with 5 participants (aged 23-26) who are current sharing a laundry room in dormitory and familiar with android applications. In the beginning, we introduced to the participants about a new laundry booking systems and wanted to get feedback from them. They had nothing about the idea of gamification and the problem we are trying to solve by applying it. They then were asked to try using all the function of {\toolname} and say whatever came into their mind (\emph{Think aloud technique}). The purpose of this step was to test usability and help the participants get used to the application. After that we explained in detail each function of the application and go through the questionnaires aiming at understanding whether {\toolname} interests them and get them change their laundry habits. For each question, the participant had to give a score (from 1 - totally disagree to 8 - totally agree) and explain their thinking. Table \ref{tab:table1} showed the questions and average scores from the participants.
\begin{table*}
  \centering
  \begin{tabular}{c|l|r}
    % \toprule
     {\textit{No.}} &{\textit{Question}} & {\textit{Avg. score (1 - 8)}} \\
    \midrule
    1&This application is easy to use. & 5.5 \\
    2&This application helps tackle problems you’re experiencing with doing laundry. & 6 \\
    3&This application helps you save time. & 6.5\\
    4&There are some missing features or information in this application that will help you with doing laundry. & 2.75 \\
    \midrule
		5&You find this application interesting and want to use this. & 6.75\\
		6&The awards/penalties parts are fun/interesting to use. & 7\\
		7&You like the Report functionality. & 5\\
		8&You are interested in earning points and advancing your ranking with this application. & 7\\
		\midrule
		9&You are willing to do your laundry in non-peak hours because you'll gain more points. & 7\\
		10&This application will change your habits of doing laundry in the long run. & 6.75
  \end{tabular}
  \caption{Questions and average score from the participants.}~\label{tab:table1}
\end{table*}
\subsubsection{Usability (Questions 1 - 4)}
All participants found it straightforward to reserve and manage their bookings because they were already familiar with other reservation systems and the interfaces are also intuitive enough to walk them through booking steps. However, they were confused with other additional elements such as points and reporting. They did not understand why they need points or what kind of reporting the application provides. This explains why the application got 5.5 in Question 1.

Almost participants are excited about \emph{Observation} and \emph{Statistics} ideas. They thought these functions are very informative and helpful in saving time and planing for doing laundry. They all thought that this application is what they really want at the moment. 
\subsubsection{User Engagement (Questions 5 - 8)}
All of the participants found this application really fun to use. One participate reflected on the usefulness of the service: \textit{``This application is fun and fare. It not only encourages users to do laundry but also punishes them if they use the machines for too long''}. Another one says:\textit{``I really like this application because of its purpose of balancing the number of demands of washing machine users. It is the best feature of this application.''}.

Most of them felt interested in awards/penalties. They thought it is reasonable to apply penalties on overuse. However, there are different opinions on reporting. While some participants would be willing to report other user even if they would not awarded points, the others disagreed. One participant suggested changes: \textit{``I don't feel good when reporting someone for points. I'd rather be reported by a machine. I'll feel uncomfortable knowing someone has just reported me.''} 

All participants were completely sure to involve themselves in earning points and advancing level because of their benefits. Furthermore, some participants believed the application could also have social impact with leader board and reporting. One says: \textit{``If my friends do care about their ranks on leader board, I also want to join them. It would be much more fun if the system allows me to share status about my progress such as: `Just got 500 points for getting to level 5' or `Being reported again!' ''}. Another one also suggested \textit{``Leader board doesn't have much influence on me. I really don't care if I get to higher or lower rank. I would only worry if they publish information about persons who usually got reported.''}
\subsubsection{Behavior Change (Questions 9 - 10)}
All of the participants thought they would obviously change their behaviors when using this application because of the benefits it brings to them. Firstly, earning points helps them save money when using the machine. Furthermore, the application brings them into a playful and competitive environment. One participant shared: \textit{``This is absolutely a fair competition. My schedule is very flexible. I will surly change my laundry habit in exchange of saving money and having fun''.}