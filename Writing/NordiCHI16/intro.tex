Reservation systems are software systems that store and retrieve information about services or facilities and conduct transactions for booking them \cite{wiki:CRS}. These systems are most commonly used by airlines or travel agencies to book flight tickets, hotel rooms and lodging facilities. In recent years, there has been a growing trend in sharing ownership of cars, bikes or some other infrequently used assets as a way to avoid paying for such rarely used facilities. In fact, shared-use vehicle systems \cite{barth2002shared, nextbike} have attracted a great deal of interest due to their benefits to the users and environment. Similar to reservation systems, these shared asset management systems not only keep track of resources status and people who are using them but also resolve common problems related to scheduling such as: double booking or over booking. However, using shared assets involves more than just simply observation and reservation; it is enmeshed in daily activities of users; which constitutes a fluctuation in demand of a particular service or asset. For example, there would be an enormous demand on hotel rooms and airline tickets in holiday seasons; shared vehicles only show their effectiveness before or after working hours. Therefore, the facilities are not used at their highest capacities while the waiting lists are lengthened during peak hours. To keep the number of demands steady, we propose an approach which gamifies traditional reservation systems.

\emph{Gamification}, application of game elements and game principles into non-game contexts to support user engagement \cite{deterding2011game, hamari2014does}, has been proved to improve service use such as increasing social interaction or quality and productivity of the actions \cite{hamari2014does}. In this paper, we aim at understanding how gamification might enhance user experience and change their behaviors in using shared facilities. For evaluation, we present the design and deployment of {\toolname} which is a gamified version of the laundry booking system. Findings from interviewing the students who are currently using a shared laundry room in dormitory offer evidence that our application has attracted much interest of the user and promised to adjust their laundry habits.