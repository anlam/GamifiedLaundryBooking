Laundry practices are not just washing and drying clothes; they are effected by the ordering of
our daily routines \cite{shove2010beyond} and other occasional factors external to laundry itself such as running out of clothes. In order to understand laundry habits and typical use scenarios of shared washing machine, we conducted user surveys and a pilot workshop involving students in the dormitory to collect information about their experiences of using shared laundry room and thinking toward an ideal management system.

It is not very surprising that the students have issues when using the communal laundry room which has a limited number of washing machines. Firstly, to do the laundry, students usually periodically go to the laundry room to check for available machines. During weekend, it could take them hours waiting for their turns. Secondly, the wash times are variable and unpredictable; machines automatically adjust their wash cycle duration according to factors such as: clothes weight, water temperature or pressure, etc. Therefore, students also have to check for the finished jobs. Sometimes, forgetting to pick up the clothes could get them left out by other student. However, these situations only happen commonly during peak hours when the vast majority of students come and do their laundry at the same time whereas the washing machines are being left unused at other times; resulting in an inefficient usage of the machines.

With those aforementioned issues of an example of ``the tragedy of the commons'' at the dormitory, the participants in the workshop also come up with some requirements for a management system. Particularly, they are looking forward to a system that allows them to reserve laundry time-slot in advance and remind them for upcoming booking or finished job. Some students do not always remember to book a time-slot, they want the system to allow them to reserve the machine at the laundry room or provide information about available machine right away without having to go to the laundry room.

Based on the collected information and envisioning, we design and prototype {\toolname}, a laundry booking system which affords:
\begin{itemize}
\item Reserving and managing time-slot on mobile phone or at the laundry room.
\item Notifying the users for upcoming bookings, finished jobs.
\item Real-time status observation of the machines.
\item Display of statistics about usage history of the machines.
\end{itemize}