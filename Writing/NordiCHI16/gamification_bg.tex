Gamification has been defined as the use of \emph{game-design elements} for \emph{non-game context} to motivate and improve user activity and retention \cite{deterding2011game, hamari2014does}. Following the success of the location-based services \emph{Foursquare}, the idea of gamification has become a trending topic in interaction design and digital marketing \cite{deterding2011game} in recent years. By using game elements as motivational affordances, gamified applications have showed their effectiveness in producing desired \emph{psychological} (e.g., user experience, engagement, fun, etc.) and \emph{behavioral outcomes} (e.g, participation, performance, productivity, etc.) \cite{deterding2011game}. Currently, there is no specified collection of game elements which could be used in gamified systems; it depends on the intentional purpose and the non-game context. Some commonly used elements are: points, level and progression, awards, goal/challenge, badges and leader-board.

There is no doubt that gamification provides effective support for various type of industries. One typical example of gamification  is \emph{Nike+}, a mobile application developed by \emph{Nike}, which successfully motivates people to run. By adding new elements to running, Nike has made running become more fun. With Nike+, users can connect, challenge, cheer and motivate their friends or running buddies around the world. \emph{Xbox Live} is another success example which uses scores, avatars and challenges to involve users into new games. With Xbox Live, \emph{Microsoft} has completely changed console gaming experience for everyone.
 
With the success stories of gamification in motivating users, we hypothesize that this concept is also beneficial to improving aforementioned situation in current reservation systems.