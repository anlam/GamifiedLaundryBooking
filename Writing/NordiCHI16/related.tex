%The idea of using game elements for reservation systems is not new. 

%The idea of a system that affords reservation, observation and notification is not new. It has been widely used in transportation, hotel or even entertainment industries for years. Nowadays, every online reservation systems allows users to reserve services in advance, manage bookings and remind them whenever the time coming. 
There has been some works putting effort on balancing reservation system requests. In \cite{gauld2000solving}, the authors proposed using priority criteria and access threshold in order to remove waiting list of surgical and medical procedures. Edara \emph{et al.} presented Highway Space Inventory Control System, a booking system for highway trip that determine whether to accept or reject a reservation based on a pre-defined demand in order to optimize the highway allocations for different traffic scenarios \cite{edara2008model}. 

There are also works applying game elements to reservation systems. Concur, a provider of spend management solutions and services, used challenges (price-to-beat) and points to encourage users book trips at lower prices \cite{concur}. BCD Travel \cite{bcd}, a provider of global corporate travel management, used rewards as an incentive to focuses users on loyalty programs (hotel, airplane) and making smart choices.
%such as introducing the concept of ``Reward Pool'' which provided its users with incentives to improve its utilization \cite{winand2006methods}
%Wynbox[4] focuses on increasing number of bookings.


%For example, today most travel reservation systems  award money to the users as an incentive for reserving unpopular time-slots.  

%All these solutions are only adaptable for a particular situation and could not be reused at all. In this project, we extend the idea of reservation system toward gamification in order to not only improve user experience but also address the optimization problem.